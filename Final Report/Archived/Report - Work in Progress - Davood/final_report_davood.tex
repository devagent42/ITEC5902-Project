\documentclass[conference]{IEEEtran}
\IEEEoverridecommandlockouts
% The preceding line is only needed to identify funding in the first footnote. If that is unneeded, please comment it out.
\usepackage{cite}
\usepackage{amsmath,amssymb,amsfonts}
\usepackage{algorithmic}
\usepackage{graphicx}
\usepackage{textcomp}
\usepackage{xcolor}
\def\BibTeX{{\rm B\kern-.05em{\sc i\kern-.025em b}\kern-.08em
    T\kern-.1667em\lower.7ex\hbox{E}\kern-.125emX}}
\begin{document}

\title{Time Series Classification \\ ITEC5920 Final Project Report}

\author{\IEEEauthorblockN{Georges Ankenmann}
\IEEEauthorblockA{\textit{School of Information Technology} \\
\textit{Carleton University}\\
Ottawa, Canada \\
GeorgesAnkenmann@cmail.carleton.ca}
\and
\IEEEauthorblockN{Davood Akhavannasab}
\IEEEauthorblockA{\textit{School of Information Technology} \\
\textit{Carleton University}\\
Ottawa, Canada \\
DavoodAkhavannasab@cmail.carleton.ca}

}

\maketitle

\begin{abstract}
The abstract.
\end{abstract}

\begin{IEEEkeywords}
deep learning, machine learning, time, series, classification, neural network, cnn, lstm, mlp, vgg, convolutional neural network, multilayer perceptron, long short-term memory, very deep convolutional networks, n-beats
\end{IEEEkeywords}

\section{Introduction}
Nowadays, sensors can be seen in many devices such as smartphones, automobiles, and industrial instruments. Also, sensor information plays a critical role in a wide range of industries from health to safety. This information usually comes in a time-series format can be helpful in different fields such as disaster prevention, human health alerts, and air quality control by measurement interpretation. Some sensor information may contain noise (usually environmentally). We can find patterns by classifying the sensor information. The most common challenging issues of classification application are when we face imbalanced data and lack of training data. For example, if you consider a time series data set containing vibration data of a bridge, we will face a lack of earthquake information while it is a rare phenomenon. In several related works, CNN (Convolutional Neural Network), Multilayer Perceptrons (MLP), and LSTM (Long Short Term Memory network) have succeeded in time-series classification problems.
\section{Research and Literature Review}
This section contains literature for deep learning for time series classification. 

\cite{b8} has proposed a model for sensor classificatoin by encoding time series data into colored images (two-dimentional). Then they did image concatenation to image feature identity and then did image classification by Convolutional Neural Network (CNN). Angular Difference Field (GADF), Gramian Angular Summation Field (GASF),  and Markov Transition Field (MTF) has been utilized for time series data transformation into images. In fact they have used these functions to prepare multivariate sensor data to obtain two-dimensional images. They have proved that their proposed model improved the classification results in comparison with other deep learning classification methods. They have used ECG and Wafer datasets in order to evaluate their model. 
\par \cite{ramakrishnan2018network} have proposed an outstanding model for network trafic classification. Their work contains three tasks for predicting future network volume traffic, future packet protocol and future packet distribution through Recurrent Neural Network(RNN), Long Short Term Memory(LSTM) and Gated Recurrent Units(GRU). In the second task(future packet protocol prediction), they have focused on layer seven protocols such as SNMP, and HTTPS. As a matter of fact, they have considered some network packet features such as packet legnth, packet header information, and statistical information such as source IP and destination port as input and have considered packet protocol as output. They have evaluated their proposed models by GEANT and Abilene public datasets. Also, they have collected real networking data from network nodes by themselves. The results show that RNN and LSTM had amazing improvements in network traffic prediction in comparison with other proposed models such as Autoregressive Integrated Moving Average (ARIMA).  

\par \cite{shen2021accurate} defines a DApp fingerprinting technique utilizing Graph Neural Networks (GNN) to identify different DApps by observing the user traffic stream. Their proposal consists of an information graph structure named Traffic Interaction Graph (TIG). TIG represents each traffic flow, comprising of packets resulting from client-server communications.  The main advantage of TIG is that it shows much data like the original stream, such as packet length, packet direction, and packet order. They proposed a GNN-base classifier that extracts input TIG's features to recognize graph structures.

\par Several works have focused on online shopping prediction by deep learning. \cite{hidasi2015session} have used RNN on recommender systems, while these systems have one common issue to make decisions based on short session-based Information and usually do not track the session information based on customer user IDs. They have utilized two datasets. The first one belongs to RecSys Challenge 2015, and the second one has been collected from an online service platform. They have eliminated 1-click sessions and combined the buyer's and clickers' Information to prepare the dataset.  They have tried to focus on classifying user intent to make a purchase. Their results show that their work has remarkably improvemened recommender systems by applying RNNs. The authors of \cite{lang2017understanding} have utilized RNN for customer behaviour prediction using clickstream information and the have proved RNN has better results than Logistic Regression. Also, they have admitted that the only disadvantage of RNNs is longer training time, but they mentioned the point that in RNNs, we do not need much feature engineering. \cite{toth2017predicting} have utilized LSTM RNNs(emphasizing single RNNs architectures) for shopping behaviour prediction using clickstream information.  

\par Several researchers such as \cite{croda2019sales} and \cite{elmasdotter2018comparative} have utilized Neural Networks to optimize inventory management systems through sales prediction. Suppose companies know about the number of their future sales. In that case, they can optimize their inventory management system. \cite{croda2019sales} have implemented MPNN(Multilayer Perceptron Neural Network) to predict future sales to anticipate the required space of the company warehouse in the future. They have used sales information of a company for one year to predict future sales and see how many warehouses should be constructed in the future. They have admitted that MPNN does not work precisely in large time-series datasets for behaviour prediction. However, still, MPNN outperforms traditional machine learning methods in small datasets for non-linear behaviour prediction. The have utilized RMSE in model evaluation. \cite{elmasdotter2018comparative} have used LSTM and ARIMA to predict grocery sales and then optimize the inventory system to reduce the number of waste foods. They have trained the models through sales information of a chain grocery company from 2013 to 2017. They have proved that LSTM has better results in comparison with ARIMA in predicting the coming seven days of given data. RMSE and MAE have been used for model evaluation. 

\section{Methodology}

\section{Results}

\section{Discussion}

\section{Contribution}
Both group members contributed equally to this project. We leveraged each others strengths to allow for an effective and concentrated group effort.
\begin{thebibliography}{00}
\bibitem{b1} A. Shrestha and . J. Dang, "Deep Learning-Based Real-Time Auto Classification of Smartphone Measured Bridge Vibration Data," 2020. 
\bibitem{b2} D. P. Francis, M. Laustsen and H. Babamoradi, "Classification of colorimetric sensor data using time series," 

\bibitem{b8} Yang, Chao-Lung and Chen, Zhi-Xuan and Yang, Chen-Yi, Sensor classification using convolutional neural network by encoding multivariate time series as two-dimensional colored images, vol. 20. MDPI, 2019, pp.168.

\bibitem{ramakrishnan2018network} Ramakrishnan, Nipun and Soni, Tarun, Network traffic prediction using recurrent neural networks, 2018 17th IEEE International Conference on Machine Learning and Applications (ICMLA), 2018, pp.187-193.

\bibitem{hidasi2015session} Hidasi, Bal{\'a}zs and Karatzoglou, Alexandros and Baltrunas, Linas and Tikk, Domonkos, Session-based recommendations with recurrent neural networks, arXiv preprint arXiv:1511.06939, 2015

\bibitem{lang2017understanding} Lang, Tobias and Rettenmeier, Matthias, Understanding consumer behavior with recurrent neural networks, Workshop on Machine Learning Methods for Recommender Systems, 2017
 
\bibitem{toth2017predicting} Toth, Arthur and Tan, Louis and Di Fabbrizio, Giuseppe and Datta, Ankur, Predicting shopping behavior with mixture of RNNs, eCOM@ SIGIR, 2017

\bibitem{croda2019sales} Croda, Rosa Mar{\'\i}a Cant{\'o}n and Romero, Dami{\'a}n Emilio Gibaja and Morales, Santiago-Omar Caballero, Sales prediction through neural networks for a small dataset, Vol. 5, NO. 4, pp. 35-41, UNIR-Universidad Internacional de La Rioja, 2019

\bibitem{elmasdotter2018comparative} Elmasdotter, Ajla and Nystr{\"o}mer, Carl, A comparative study between LSTM and ARIMA for sales forecasting in retail, 2018

\bibitem{shrestha2020deep} Shrestha, Ashish and Dang, Ji, Deep learning-based real-time auto classification of smartphone measured bridge vibration data,Vol. 20, NO. 9, Multidisciplinary Digital Publishing Institute, 2020

\end{thebibliography}


\end{document}
